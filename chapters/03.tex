\sub{Present long form conjugation}
    {
    \begin{tabular}{ |r l l| } 
        \hline
                        & \textit{ru}-verb  & \textit{u}-verb \\ 
        Dictionary form & \ruby{食}{た}べる & \ruby{行}{い}く \\
        Affirmative     & 食べます          & 行きます \\
        Negative        & 食べません        & 行きません \\
        \hline
    \end{tabular}
    
    Note: Don't forget irregular verbs such as する and くる.
    }

\sub{Particles}
    {
    \begin{itemize}
        \item は - topic of sentence.
        \item を - object of verb.
        \item で - where event happened.
        \item に - specific date/time or specific destination.
        \item へ - general direction (north, south, etc.) or far away location.
    \end{itemize}
    }

\sub{〜ませんか}
    {
    Use to extend an invitation in polite manner.
    
    テニスし\hl{ませんか}。\textit{\hl{Would you like to} play tennis?}
    }

\sub{Frequency Adverbs}
    {
    \begin{tabular}{|l l|}
        \hline
        \ruby{毎日}{まいにち} & everyday \\
        \hline
        いつも                & always \\
        \hline
        たいてい              & usually \\
        \hline
        よく                  & often \\
        \hline
        ときどき              & sometimes \\
        \hline
        あまり                & not very often \\
        \hline
        ぜんぜん              & never \\
        \hline
    \end{tabular} 
    
    Note: あまり and ぜんぜん must be used in conjunction with a negative form of a verb. i.e.
    
    テレビを\hl{あまり}\ruby{見}{み}\underline{ません}。\textit{I \underline{don't} watch TV \hl{very often}.}
    }
