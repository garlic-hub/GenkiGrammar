\sub{Xがあります/います}
    {
    \begin{center}
        \begin{tabular}{|rc|l|}
        \hline
                   &  thing が \hl{あります} & \hspace{0mm} \\
        (place に) &                         & \textit{There is/are...} \\
                   &  person が \hl{います}  & \hspace{0mm} \\
        \hline
        \end{tabular}
    \end{center}
    }

\sub{Describing where things are}
    {
    \begin{center}
        \begin{tabular}{|ccc|cl|}
            \hline
                           & あそこ &       &               & over there. \\
            マクドナルドは & そこ   & です。& McDonald's is & right there by you. \\
                           & ここ   &       &               & right here. \\
            \hline
        \end{tabular} 
    \end{center}
    
    \begin{center}
        \begin{tabular}{|ccc|ccc|}
            \hline
                   & \underline{location words} &&&& \hspace{0mm} \\
                   & みぎ          &&      & to the right of     & \hspace{0mm} \\
                   & ひだり        &&      & to the left of      & \hspace{0mm} \\
                   & まえ          &&      & in front of         & \hspace{0mm} \\
            XはYの & なか & です。  & X is & inside              & Y. \\
                   & うえ          &&      & on/above            & \hspace{0mm} \\
                   & した          &&      & under/beneath       & \hspace{0mm} \\
                   & ちかく        &&      & near                & \hspace{0mm} \\
                   & となり        &&      & next to             & \hspace{0mm} \\
            \hline
        \end{tabular}
    \end{center}
    
    XはYとZのあいだです。X is between Y and Z.
    }

\sub{Past tense of です}
    {
    \begin{center}
        \begin{tabular}{|lll|}
            \hline
                          & affirmative & negative \\
            present tense &  〜です     & 〜じゃないです \\
            past tense    &  〜でした   & 〜じゃなかったです \\
            \hline
        \end{tabular}
    \end{center}
    }

\sub{Past long form conjugation}
    {
    \begin{center}
        \begin{tabular}{|rll|} 
            \hline
                          & affirmative & negative \\
            present tense & 〜ます      & 〜ません \\
            past tense    & 〜まして    & 〜ませんでした \\
            \hline
        \end{tabular}
    \end{center}
    }

\sub{もas also/too}
    {
    \ruby{犬}{いぬ}が\ruby{好}{す}きです。\textit{I like dogs.}
    
    \ruby{私}{わたし}\hl{も}!\textit{Me \hl{too}!}
    
    \sectionSplit
    
    \ruby{東京}{とうきょう}に\ruby{行}{い}きました。\ruby{大阪}{おおさか}\hl{にも}\ruby{行}{い}きました。
    \textit{I went to Tokyo. I went to Osaka too.}
    }

\sub{Time duration}
    {
    How to express time duration in hours:
    \begin{center}
        \begin{tabular}{|ll|}
        \hline
            X hours            & X\ruby{時間}{じかん} \\
            X and a half hours & X\ruby{時間半}{じかんはん} \\
            Around X hours     & X\ruby{時間}{じかん}ぐらい \\ 
            Around X and a half hours & X\ruby{時間半}{じかんはん}ぐらい \\
        \hline
        \end{tabular}
    \end{center}
    
    きのう\ruby{七\hl{時間}}{しちじかん}ねました。\textit{I slept 7 hours yesterday.}
    
    メアリーさんはたけしさんを\ruby{一\hl{時間半}}{いちじかんはん}\ruby{待}{ま}ちました。\textit{Mary waited an hour and a half for Takeshi.}
    
    \ruby{二\hl{時間半}}{にじかんはん}ぐらい\ruby{出}{で}かけました。\textit{I went out for about 2 and a half hours.}
    }
    
\sub{たくさん}
    {
    Expressions of quantity can be placed before the noun or after the particle in Japanese.
    
    The following sentences are equivalent:
    \begin{itemize}
        \item \hl{たくさん}\ruby{天}{てん}ぷらを\ruby{食}{た}べました。
        \item \ruby{天}{てん}ぷらを\hl{たくさん}\ruby{食}{た}べました。
    \end{itemize}
    \textit{I ate \hl{a lot} of tempura.}
    }

\sub{と particle}
    {
    \begin{itemize}
        \item Combines nouns. Can be chained together i.e. AとBとC
        \item Describes whom you do something with.
    \end{itemize}
    
    \ruby{日本語}{にほんご}\hl{と}\ruby{英語}{えいご}を\ruby{話}{はな}します。\textit{I speak Japanese \hl{and} English.}
    
    メアリーさんはスーさん\hl{と}\ruby{日本}{日本}に\ruby{行}{い}きました。\textit{Mary went to Japan \hl{with} Sue.}
    }