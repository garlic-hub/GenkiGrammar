\sub{Xがあります/います}
    {
    \begin{tabular}{r|c|l}
               &  thing が \hl{あります} \\
    (place に) &                         & \textit{There is/are...} \\
               &  person が \hl{います}
    \end{tabular}
    }

\sub{Describing where things are}
    {
    \begin{tabular}{c|c|cc|l}
                       & あそこ &       &               & over there. \\
        マクドナルドは & そこ   & です。& McDonald's is & right there by you. \\
                       & ここ   &       &               & right here.
    \end{tabular} 
    
    \begin{tabular}{c|c|cc|c|c}
               & location words \\
               & みぎ          &&      & to the right of \\
               & ひだり        &&      & to the left of \\
               & まえ          &&      & in front of \\
        XはYの & なか & です。  & X is & inside              & Y. \\
               & うえ          &&      & on/above \\
               & した          &&      & under/beneath \\
               & ちかく        &&      & near \\
               & となり        &&      & next to
    \end{tabular}
    
    XはYとZのあいだです。X is between Y and Z.
    }

\sub{Past tense of です}
    {
    \begin{tabular}{|lll|}
        \hline
                      & affirmative & negative \\
        present tense &  〜です     & 〜じゃないです \\
        past tense    &  〜でした   & 〜じゃなかったです \\
        \hline
    \end{tabular}
    }

\sub{Past long form conjugation}
    {
    \begin{tabular}{|rll|} 
        \hline
                      & affirmative & negative \\
        present tense & 〜ます      & 〜ません \\
        past tense    & 〜まして    & 〜ませんでした \\
        \hline
    \end{tabular}
    }

\sub{もas also/too}
    {
    \ruby{犬}{いぬ}が\ruby{好}{す}きです。\textit{I like dogs.}
    
    \ruby{私}{わたし}\hl{も}!\textit{Me \hl{too}!}
    
    \sectionSplit
    
    あなた\hl{も}アメリカ\ruby{人}{じん}ですか。\textit{Are you \hl{also} American?}
    }

\sub{Time duration}
    {
    To express time duration, use a number then 〜\ruby{時間}{じかん}.
    Append a \ruby{半}{はん} to express "and a half".
    Append a ぐらい to express "about".
    
    きのう\ruby{七\hl{時間}}{しちじかん}ねました。\textit{I slept 7 hours yesterday.}
    
    メアリーさんはたけしさんを\ruby{一\hl{時間半}}{いちじかんはん}\ruby{待}{ま}ちました。\textit{Mary waited an hour and a half for Takeshi.}
    
    \ruby{二\hl{時間半}}{にじかんはん}ぐらい\ruby{出}{で}かけました。\textit{I went out for about 2 and a half hours.}
    }
    
\sub{たくさん}
    {
    Expressions of quantity can be placed before the noun or after the particle in Japanese.
    
    The following sentences are equivalent:
    \begin{itemize}
        \item \hl{たくさん}\ruby{天}{てん}ぷらを\ruby{食}{た}べました。
        \item \ruby{天}{てん}ぷらを\hl{たくさん}\ruby{食}{た}べました。
    \end{itemize}
    \textit{I ate \hl{a lot} of tempura.}
    }

\sub{と particle}
    {
    \begin{itemize}
        \item Combines nouns. Can be chained together i.e. AとBとC
        \item Describes whom you do something with.
    \end{itemize}
    
    \ruby{日本語}{にほんご}\hl{と}\ruby{英語}{えいご}を\ruby{話}{はな}します。\textit{I speak Japanese \hl{and} English.}
    
    メアリーさんはスーさん\hl{と}\ruby{日本}{日本}に\ruby{行}{い}きました。\textit{Mary went to Japan \hl{with} Sue.}
    }