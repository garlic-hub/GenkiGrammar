\sub{Adjectives}
    {
    い-adjectives

    \begin{tabular}{|ccc|}
        \hline 
                 & affirmative          & negative \\
         present & よ\underline{い}     & よ\underline{くない} \\
                 & \textit{good}        & \textit{not good} \\
        \\
         past    & よ\underline{かった} & よ\underline{くなかった} \\
                 & \textit{was good}    & \textit{was not good} \\
        \hline 
    \end{tabular}

    な-adjectives

    \begin{tabular}{|ccc|}
        \hline 
                 & affirmative                         & negative \\
         present & \ruby{元気}{げんき}\underline{です} & 元気\underline{じゃないです} \\
                 & \textit{healthy}                    & \textit{not healthy} \\
        \\
         past    & 元気\underline{でした}              & 元気\underline{じゃなかったです} \\
                 & \textit{was healthy}                & \textit{was not healthy} \\
        \hline 
    \end{tabular}

    \ruby{天気}{てんき}は\underline{すごく}\ruby{\hl{さむかった}}{いadjective}です。
    \textit{The weather \hl{was} \underline{very} \hl{cold}}.
    }

\sub{好き(な)/きらい(な)}
    {
    How to say you like or dislike something. Be careful. These words carry a lot more power in Japanese than they do in English and you could find yourself in an awkward situation.

    ロバートさんは\ruby{日本語}{にほんご}のくらすが\ruby{\hl{好}}{す}\hl{き}です。
    \textit{Robert \hl{likes} Japanese class.}

    \ruby{山下先生}{やましたせんせい}は\ruby{魚}{さかな}が\hl{きらい}です。
    \textit{Professor Yamashita \hl{dislikes} fish.}

    Change \ruby{好}{す}き and きらい to \ruby{大好}{だいす}き and \ruby{大}{だい}きらい to express "like very much" and "hate", respectively.

    ロバートさんは\ruby{日本語}{にほんご}のくらすが\ruby{\hl{大好}}{だいす}\hl{き}です。
    \textit{Robert \hl{likes} Japanese class \hl{a lot}.}

    \ruby{山下先生}{やましたせんせい}は\ruby{魚}{さかな}が\ruby{\hl{大}}{だい}\hl{きらい}です。
    \textit{Professor Yamashita \hl{hates} fish.}

    As a noun modifier

    これは\ruby{\underline{好}}{す}\underline{きな}\ruby{本}{ほん}です。
    \textit{This is my \underline{favorite} book.}
    }

\sub{〜ましょう/〜ましょうか}
    {
    A verb conjugation that expresses "Let's do X."

   \ruby{図書館}{としょかん}で\ruby{勉強}{べんきょう}し\hl{ましょう}。 
   \textit{Let's study at the library.}

    コーヒーを\ruby{飲}{の}み\hl{ましょうか}。
    \textit{Want to get coffee?}
    }

\sub{Counting}
    {
    \begin{itemize}
        \item Numbers come after the particle in Japanese.
        \item Japanese has a concept of counters where the number suffix changes depending on what is being counted. Don't worry too much about this for now. 
    \end{itemize}

    メアリーさんは\ruby{本}{ほん}を\ruby{\hl{三}\underline{冊}}{さんさつ}\ruby{買}{か}いました。
    \textit{Mary bought \hl{three} books.}

    The underlined portion is not a noun. It is the counter for books.
    }