\sub{Present Short Form}
    {
    Short form, plain form, informal form, and direct style are all different names for the same conjugation.
    
    It's best to just review the book for this one. 
    }

\sub{Informal Speech}
    {
    When family, close friends, etc. speak it's a more casual use of the language. Short forms of verbs and adjectives are used. Other changes happen as well.
    
    Remember that if someone uses casual speech with you, that does not always mean you should use it back.
    
    It's best to just review the book for this one. 
    }

\sub{〜と思ういます/〜と言っていました}
    {
    To quote thoughts or speech, you use these clauses at the end of a sentence. とgains another use as a particle: it indicates quotation.
    
    When using とparticle to quote you must change the quoted text/speech to use short form.
    
    Sue said to Mary:
    
    \ruby{明日}{あした}テストがあります。
    \textit{There is a test tomorrow.}
    
    To quote this text Mary would say:
    
    スーさんは\ruby{明日}{あした}テストが\hl{ある}と\ruby{言}{い}っていました。
    \textit{Sue said there is a test tomorrow.}
    
    \sectionSplit
    
    To say something you think may be the case:
    
    (\ruby{私}{わたし}は)たけしさんはメアリーさんが\ruby{好}{す}き\hl{だ}と\ruby{思}{おも}います。
    \textit{I think Takeshi likes Mary.}
    
    To say something you think may not be the case:
    
    (\ruby{私}{わたし}は)たけしさんはメアリーさんが\ruby{好}{す}き\hl{じゃない}と\ruby{思}{おも}います。
    \textit{I don't think Takeshi likes Mary.}
    
    Notice we changed the quoted text to be negative. Not the \ruby{思}{おも}うto be negative.
    }

\sub{〜ないでください}
    {
    Use to ask someone to please not do something.

    \begin{center}
        \fbox{negative short form + でください\quad\quad\quad\quad\textit{Please don't\dots}}
    \end{center}

    うるさく\ruby{話}{はな}\hl{さないでください}。
    \textit{Please don't talk loudly.}

    ここで\ruby{写真}{しゃしん}を\ruby{撮}{と}\hl{らないでください}。
    \textit{Please don't take photo's here.}

    Note that this differs from section \ref{5.4} because it does not relate to rules or regulations.
    }

\sub{Turning verbs into nouns}
    {
    \begin{center}
        \fbox{short form + の}
    \end{center} 
    
    \ruby{日本語}{にほんご}を\ruby{勉強}{べんきょう}\hl{するの}が\ruby{好}{す}きです。
    \textit{I like \hl{studying} Japanese.}
    
    メアリーさんは\ruby{日本語}{にほんご}を\ruby{\hl{話}}{はな}\hl{すの}が\ruby{上手}{じょうず}です。
    \textit{Mary is good at [\hl{speaking}] Japanese.}
    }

\sub{が as subject particle}
    {
    がas a subject marking particle serves two possible purposes:
    
    \begin{itemize}
        \item Emphasizes the subject.
        \item Used with question words.
    \end{itemize}
    
    ロバートさんは\ruby{学校}{がっこう}に\ruby{行}{い}きました。
    \textit{Robert went to school.}
    
    vs
    
    
    ロバートさん\hl{が}\ruby{学校}{がっこう}に\ruby{行}{い}きました。
    \textit{Robert is the one who went to school.}
    
    \sectionSplit
    
    どのくらす\hl{が}むずかしい。
    \textit{Which class is [most] difficult?}
    
    \ruby{英語}{えいご}のくらす\hl{が}むずかしい。 
    \textit{English class.}
    }

\sub{何かand何も}
    {
    These words can be used in place of words where particles like は, が, and を are expected.
    
    \begin{tabular}{|ll|}
        \hline
        \ruby{何}{なに}か            &  something \\
        \ruby{何}{なに}か            &  anything \\
        \ruby{何}{なに}も + negative &  nothing/not anything \\
        \hline
    \end{tabular}
    
    \ruby{\hl{何}}{なに}\hl{か}\ruby{聞}{き}ましたか。
    \textit{Did you hear \hl{something}?}
    
    \ruby{今日}{きょう}\ruby{\hl{何}}{なに}\hl{か}しましたか。
    \textit{Did you do \hl{anything} today?}
    
    \ruby{\hl{何}}{なに}\hl{も}食べました。
    \textit{I did \hl{not} eat \hl{anything}.} 
    }