\sub{Past tense short form}
    {
    Refer to \ref{8.1} for present tense short form.

    It's best to just review the book for this one.
    }

\sub{Qualifying Nouns with Verbs and Adjectives}
    {
    You can use adjectives and short form verbs to qualify nouns.
    
    \begin{tabular}{|rcl|}
        \hline
        おもしろい                         &  \ruby{人}{ひと} & \textit{A person \fbox{who is interesting}} \\
        \ruby{髪}{かみ}が\ruby{長}{なが}い & 人               & \textit{A person \fbox{who has long hair}} \\
        めがねをかけている                 & 人               & \textit{A person \fbox{who wears glasses}} \\
        \ruby{猫}{ねこ}が\ruby{好}{す}きな & 人               & \textit{A person \fbox{who likes cats}} \\
        \multicolumn{1}{|c}{\textuparrow}  & \textuparrow     & \hspace{0mm} \\
        adjective/verb qualifiers          & noun             & \hspace{0mm} \\
        \hline
    \end{tabular}
    }

\sub{まだ〜ていません}
    {
    To do
    }

\sub{〜から}
    {
    Section \ref{6.6} taught us how to use 〜から at the end of a separate sentence to explain the reason or cause of something. You can also place 〜から directly into the first sentence itself.

    \begin{center}
        \fbox{
            \begin{minipage}{0.45\linewidth}
                (explanation)から、(situation)。

                \quad\quad = (situation), \textit{because} (explanation).

                \quad\quad = (explanation), \textit{therefore} (situation).
            \end{minipage}
        }
    \end{center}

    You can use either short or long form before the 〜から depending on how formal/polite you want the sentence to be.

    \ruby{雨}{あめ}が\ruby{降}{ふ}った\hl{から}、\ruby{出}{で}かけなかった。
    \textit{\hl{Because} it was raining, I didn't go out.}

    \ruby{寝}{ね}るのが\ruby{難}{むずか}しいです\hl{から}、\ruby{全部}{ぜんぶ}の\ruby{電気}{でんき}を\ruby{消}{け}してください。
    \textit{It's hard for me to sleep, \hl{so} please turn off all the lights.}
    }
