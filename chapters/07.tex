\sub{〜ている}
    {
    〜ている is nuanced in its use. Context will determine how to understand its use. It can describe all the following:
    \begin{itemize}
        \item An action in progress. Also known as progressive tense.
        
        \textit{I am eating pizza. He is driving to school. Mary is watering her plants.}
        \item The current state that someone/something is currently in.
        
        \textit{She is married. It is cloudy. They are wearing glasses.}
        \item Habitual or repeated actions.
        
        \textit{Takeshi studies everyday. John bikes to school sometimes.}
    \end{itemize}

    Note that the いる in 〜ている can be conjugated like any るverb.

    ピザを\ruby{食}{た}べ\hl{ています}。
    \textit{I \hl{am eating} pizza.}

    ピザを\ruby{食}{た}べ\hl{ていました}。
    \textit{I \hl{was eating} pizza.}

    \ruby{山下先生}{やましたせんせい}は\ruby{結婚}{けっこん}し\hl{ています}。
    \textit{Professor Yamashita \hl{is married}.}

    \ruby{家族}{かぞく}は\ruby{東京}{とうきょう}に\ruby{住}{す}\hl{んでいます}。
    \textit{My family \hl{lives} in Tokyo.}

    メアリーさんはTシャツを\ruby{着}{き}\hl{ています}。
    \textit{Mary \hl{is wearing} a T-shirt.}

    たけしさんは\ruby{毎日}{まいにち}\ruby{本}{ほん}を\ruby{読}{よ}\hl{んでいます}。
    \textit{Takeshi reads a book everyday.}

    \sectionSplit

    An example with more ambiguous meanings:

    \ruby{英語}{えいご}を\ruby{教}{おし}え\hl{ています}。

    \textit{I am teaching English.} (As in at this current moment in time.)

    OR

    \textit{I teach English.} (In general. Maybe as a career or as a hobby.)
    }

\sub{Describing aspects about people/objects}
    {
    Lets say you wanted to describe an aspect of a man such as them having red hair. There are two ways to do this:

    \begin{itemize}
        \item That man's hair is red.
        \item That man has red hair.
    \end{itemize}

    Both are valid, but the later is more natural in Japanese.

    \sectionSplit

    Aさんはnounがadjectiveです。

    田中さんの\ruby{髪}{かみ}は\ruby{長}{なが}いです。
    \textit{Tanaka's hair is long.}

    vs

    田中さんは\ruby{髪}{かみ}が\ruby{長}{なが}いです。
    \textit{Tanaka has long hair.} (This is likely what you want.)
    }

\sub{Te-form for joining adjectives and nouns}
    {
    Use てform with adjectives and nouns to form more complex sentences.

    \begin{center}
        \begin{tabular}{|rll|}
            \hline
                         & dictionary                    & てform \\
            いadjectives & \ruby{安}{やす}い             & \ruby{安}{やす}\underline{くて} \\
            なadjectives & \ruby{元気}{げんき}           & \ruby{元気}{げんき}\underline{で} \\
            nouns        & \ruby{日本人}{にほんじん}です & \ruby{日本人}{にほんじん}\underline{で} \\
            \hline
        \end{tabular}
    \end{center}
    
    To combine strings of adjectives or nouns, apply てform to each adjective or noun except the last one.

    そのレストランは\ruby{安}{やす}\hl{くて}おいしいです。
    \textit{That restaurant is cheap and delicious.}

    ホテルはきれい\hl{で}\ruby{大}{おお}き\hl{くて}\ruby{高}{たか}いです。
    \textit{The hotel is pretty, large, and expensive.}

    \ruby{山本}{やまもと}さんは\ruby{日本人}{にほんじん}\hl{で}\ruby{背}{せ}が\ruby{低}{ひく}\hl{くて}\ruby{二十五歳}{にじゅうごさい}です。
    \textit{Yamamoto is Japanese, short, and 25 years old.}
    }

\sub{verb stem + に行く}
    {
    Use to describe going somewhere to do something.

    \begin{center}
        \begin{tabular}{|ccc|}
            \hline
                        & に & \\
            destination &    & verb stemに movement verb \\
                        & へ & \\
            \hline
        \end{tabular}
    \end{center}

    スーパーに\ruby{肉}{にく}を\ruby{\hl{買}}{か}\hl{いに}\ruby{行}{い}きました。
    \textit{I went to the super market to buy meat.}

    \ruby{\hl{料理}}{りょうり}\hl{しに}\ruby{帰}{かえ}っています。
    \textit{I'm going home to cook.} (帰る without a destination means going home.)
    }

\sub{Counting people}
    {
    Use \ruby{人}{にん} as the counter for people. Notice "one person" and "two people" is irregular.
    
    \begin{tabular}{|llcll|}
        \hline
        ひとり(一人)  & one person   & \hspace{5mm} & ろくにん(六人)          & six people \\
        ふたり(二人)  & two people   &              & しちにん・ななにん(七人)& seven people \\
        さんにん(三人)& three people &              & はちにん(八人)          & eight people \\
        よにん(四人)  & four people  &              & きゅうにん(九人)        & nine people \\
        ごにん(五人)  & five people  &              & Xにん(X人)              & X people \\
        \hline
    \end{tabular}
    }
