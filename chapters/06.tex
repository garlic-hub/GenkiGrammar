\sub{Te-form}
    {
    A \textit{very important} form of conjugation in Japanese. Make sure to know it well. Over time you will begin to naturally apply these, but beginners should just commit the left 3 tables to memory.

   \begin{tabular}{lr}
        \begin{minipage}[b]{0.45\linewidth}
        るverbs

        \begin{tabular}{|ccc|}
            \hline
            る & $\rightarrow$ & て \\
            \hline
        \end{tabular}

        \vspace{1mm}
        うverbs

        \begin{tabular}{|ccc|}
            \hline
            う つ る & $\rightarrow$ & って \\
            む ぶ ぬ & $\rightarrow$ & んで \\ 
            く         & $\rightarrow$ & いて \\ 
            ぐ         & $\rightarrow$ & いで \\ 
            す         & $\rightarrow$ & して \\ 
            \hline
        \end{tabular}

        \vspace{1mm}
        Irregular Verbs

        (notice how common these verbs are)

        \begin{tabular}{|rcl|}
            \hline
            する & $\rightarrow$ & して \\
            \ruby{来}{き}る & $\rightarrow$ & \ruby{来}{き}て \\
            \ruby{行}{い}く & $\rightarrow$ & \ruby{行}{い}って \\
            \hline
        \end{tabular}
        \end{minipage}
    &
        \begin{minipage}[b]{0.45\linewidth}
        Examples with each verb ending

        \begin{tabular}{|rcl|}
            \hline
                              & る            & \hspace{1mm} \\
            \ruby{食}{た}べる & $\rightarrow$ & \ruby{食}{た}べて \\
            \ruby{見}{み}る   & $\rightarrow$ & \ruby{見}{み}て \\
            \hline
                              & う            & \hspace{1mm} \\
            \ruby{会}{あ}う   & $\rightarrow$ & \ruby{合}{あ}って \\
            \ruby{待}{ま}つ   & $\rightarrow$ & \ruby{待}{ま}って \\
            とる              & $\rightarrow$ & とって \\
            \ruby{読}{よ}む   & $\rightarrow$ & \ruby{読}{よ}んで \\
            \ruby{遊}{あそ}ぶ & $\rightarrow$ & \ruby{遊}{あそ}んで \\
            \ruby{死}{し}ね   & $\rightarrow$ & \ruby{死}{し}んで \\
            \ruby{書}{か}く   & $\rightarrow$ & \ruby{書}{か}いて \\
            \ruby{泳}{およ}ぐ & $\rightarrow$ & \ruby{泳}{およ}いで \\
            \ruby{話}{はな}す & $\rightarrow$ & \ruby{話}{はな}して \\
            \hline
        \end{tabular}
        \end{minipage}
   \end{tabular} 
    }

\sub{〜てください}
    {
    Use to make a polite request to another person.
    
    \begin{center}
        \fbox{てform + ください}
    \end{center}
    
    \ruby{窓}{まど}を\ruby{閉}{し}め\hl{てください}。
    \textit{Please close the window.}
    }

\sub{〜てもいいですか}
    {
    Use to ask permission for something.
    
    \begin{center}
        \fbox{てform + もいいですか。}
    \end{center}

    お\ruby{手洗}{てあら}いに\ruby{行}{い}っ\hl{てもいいですか}。
    \textit{May I go to the bathroom?}
    }

\sub{〜てはいけません}
    {
    Use to express "you must not do\dots". This is a strong statement and is used to express rules and regulations. Should not used to deny a request someone makes unless in a position of authority.
    
    \begin{center}
        \fbox{てform + はいけません。}
    \end{center}
    
    Important note: The は is a particle, so this is pronounced \textit{wa ikemasen}.
    
    ここで\ruby{吸}{す}っ\hl{てはいけません}。
    \textit{No smoking here.}
    
    \sectionSplit
     
    先生。\ruby{写真}{しゃしん}をとってもいいですか。
    \textit{Teacher. May I take photos.}
    
    いいえ、とっ\hl{てはいけません}。
    \textit{Sorry, you may not [take photos].} (Due to the rules).
    }

\sub{Combining verbs}
    {
    Use てform to combine two or more verbs to describe a sequence of actions. てform does not convey a tense, however the verbs of the sentence will all follow the tense of the final verb in the sentence.
    
    \ruby{七時}{しちじ}に\ruby{起}{お}き\hl{て}、\ruby{学校}{がっこう}に\ruby{行}{い}きます。
    \textit{I will get up at 7 and [then] go to school.}
    
    \ruby{七時}{しちじ}に\ruby{起}{お}き\hl{て}、\ruby{学校}{がっこう}に\ruby{行}{い}きました。
    \textit{I got up at 7 and [then] went to school.}
    
    \ruby{七時}{しちじ}に\ruby{起}{お}き\hl{て}、\ruby{朝}{あさ}ご\ruby{飯}{はん}を食べ\hl{て}、\ruby{学校}{がっこう}に\ruby{行}{い}きました。
    \textit{I got up at 7, [then] ate breakfast, and [then] went to school.}
    }

\sub{〜から}
    {
    A sentence that ends in から explains the reason or cause of something.
    
    \begin{center}
        \fbox{\textit{(situation). (explanation)}から。}
    \end{center}
    
    \ruby{行}{い}きません。\ruby{寒}{さむ}いです\hl{から}。
    \textit{I'm not going. [Because] it's cold outside.}
    
    バスに\ruby{乗}{の}りましょう。タクシーは\ruby{高}{たか}いです\hl{から}。
    \textit{Let's take the bus. [Because] taxis are expensive.}
    }

\sub{〜ましょうか}
    {
    Used to offer assistance in terms of "let me do that for you."
    
    You see someone with their hands full struggling to pick up something: 
    やり\hl{ましょうか}。
    
    \textit{Let me do that for you.}
    
    \sectionSplit
    
    You see someone carrying a heavy bag: \ruby{荷物}{にもつ}を\ruby{持}{も}ち\hl{ましょうか}。
    
    \textit{Let me carry that [bag/luggage] for you.}
    
    We also learned of 〜ましょうか in section \ref{5.3}.
    }