\sub{これ・それ・あれ・どれ}
    {
    this・that・that over there・which?

    \hl{これ}は\ruby{私}{わたし}の\ruby{本}{ほん}です。
    \textit{\hl{This} is my book.}
    }

\sub{この・その・あの・どの + noun}
    {
    this・that・that over there・which?
    
    These are different from section \ref{2.1} above because a noun must follow.

    \hl{この}\ruby{\underline{本}}{ほん}は\ruby{五百円}{ごひゃくえん}です。
    \textit{\hl{This} \underline{book} is 500 yen.}
    }
    
\sub{ここ・そこ・あそこ・どこ}
    {
    here・there・over there・where?

    レストランは\hl{どこ}ですか?
    \textit{hl{Where} is the restaurant?}

    \hl{あそこ}です。
    \textit{It's \hl{over there}.}
    }

\sub{だれのnoun}
    {
    これは\hl{だれの}かばんですか。
    \textit{\hl{Whose} bag is this?}

    それはメアリーさんのかばんです。
    \textit{That is Mary's bag.}
    }
    
\sub{Nounも}
    {
    メアリーさんは学生です。
    \textit{Mary is a student.}

    \ruby{私}{わたし}\hl{も}\ruby{学生}{がくせい}です。
    \textit{I am \hl{also} a student.}
    }
    
\sub{Nounじゃないです}
    {
    メアリーさんは\ruby{先生}{せんせい}\hl{じゃないです}。
    \textit{Mary \hl{is not} a teacher.} 
    }

\sub{〜ね/〜よ}
    {
    Sentence suffixes used to help convey certain meanings or tones.
    \begin{enumerate}
        \item そうです\hl{ね}。
        \item おすしが\ruby{大好}{だいす}きです\hl{よ}。
    \end{enumerate}
    }
